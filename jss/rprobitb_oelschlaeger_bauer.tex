\documentclass[article]{jss}

%% -- LaTeX packages and custom commands ---------------------------------------

%% recommended packages
\usepackage{thumbpdf,lmodern}

%% additional packages
\usepackage{amssymb,amsmath}

%% new custom commands
\newcommand{\class}[1]{`\code{#1}'}
\newcommand{\fct}[1]{\code{#1()}}

%% For Sweave-based articles about R packages:
%% need no \usepackage{Sweave}



%% -- Article metainformation (author, title, ...) -----------------------------

%% - \author{} with primary affiliation
%% - \Plainauthor{} without affiliations
%% - Separate authors by \And or \AND (in \author) or by comma (in \Plainauthor).
%% - \AND starts a new line, \And does not.
\author{Lennart Oelschl\"ager \And Dietmar Bauer}
\Plainauthor{Lennart Oelschl\"ager, Timo Adam, Rouven Michels}

%% - \title{} in title case
%% - \Plaintitle{} without LaTeX markup (if any)
%% - \Shorttitle{} with LaTeX markup (if any), used as running title
\title{\pkg{RprobitB}: Bayes Estimation of Probit Models in \proglang{R}}
\Plaintitle{RprobitB: Bayes Estimation of Probit Models in R}
\Shorttitle{RprobitB}

%% - \Abstract{} almost as usual
\Abstract{
\pkg{RprobitB} is an \proglang{R} package for Bayes estimatino of probit models.
}

%% - \Keywords{} with LaTeX markup, at least one required
%% - \Plainkeywords{} without LaTeX markup (if necessary)
%% - Should be comma-separated and in sentence case.
\Keywords{probit models, discrete choice, \proglang{R}}
\Plainkeywords{probit models, discrete choice, R}

%% - \Address{} of at least one author
%% - May contain multiple affiliations for each author
%%   (in extra lines, separated by \emph{and}\\).
%% - May contain multiple authors for the same affiliation
%%   (in the same first line, separated by comma).
\Address{
  Lennart Oelschl\"ager\\
  Department of Business Administration and Economics\\
  Bielefeld University\\
  Postfach 10 01 31\\
  E-mail: \email{lennart.oelschlaeger@uni-bielefeld.de}
}

\begin{document}
%% I have no idea what this does. Maybe we need this in the future.
%% \SweaveOpts{concordance=TRUE}

%% -- Introduction -------------------------------------------------------------

%% - In principle "as usual".
%% - But should typically have some discussion of both _software_ and _methods_.
%% - Use \proglang{}, \pkg{}, \fct{} and \code{} markup throughout the manuscript.
%% - If such markup is in (sub)section titles, a plain text version has to be
%%   added as well.
%% - All software mentioned should be properly \cite-d.
%% - All abbreviations should be introduced.
%% - Unless the expansions of abbreviations are proper names (like "Journal
%%   of Statistical Software" above) they should be in sentence case (like
%%   "generalized linear models" below).

\section{Introduction}
\label{sec:intro}

Introduction...


%% -- Manuscript ---------------------------------------------------------------

%% - In principle "as usual" again.
%% - When using equations (e.g., {equation}, {eqnarray}, {align}, etc.
%%   avoid empty lines before and after the equation (which would signal a new
%%   paragraph.
%% - When describing longer chunks of code that are _not_ meant for execution
%%   (e.g., a function synopsis or list of arguments), the environment {Code}
%%   is recommended. Alternatively, a plain {verbatim} can also be used.
%%   (For executed code see the next section.)
%% - Tables are placed at the top of the page
%%   (\verb|[t!]|), centered (\verb|\centering|), with a caption below the table,
%%   column headers and captions in sentence style, and if possible avoiding
%%   vertical lines.

\section{The method} \label{sec:method}

Content. This is a \cite{test:21}.

%% -- Illustrations ------------------------------------------------------------

%% - Virtually all JSS manuscripts list source code along with the generated
%%   output. The style files provide dedicated environments for this.
%% - In R, the environments {Sinput} and {Soutput} - as produced by Sweave() or
%%   or knitr using the render_sweave() hook - are used (without the need to
%%   load Sweave.sty).
%% - Equivalently, {CodeInput} and {CodeOutput} can be used.
%% - The code input should use "the usual" command prompt in the respective
%%   software system.
%% - For R code, the prompt "R> " should be used with "+  " as the
%%   continuation prompt.
%% - Comments within the code chunks should be avoided - these should be made
%%   within the regular LaTeX text.
%% - Please make sure that all code is properly spaced, e.g., using
%%   \code{y = a + b * x} and \emph{not} \code{y=a+b*x}.
%% - JSS prefers when the second line of code is indented by two spaces.

\section{Illustrations} \label{sec:illustrations}

%
\begin{Schunk}
\begin{Sinput}
> x = 1
\end{Sinput}
\end{Schunk}
%


%% -- Summary/conclusions/discussion -------------------------------------------

\section{Summary and discussion} \label{sec:summary}

Content ...

%% -- Optional special unnumbered sections -------------------------------------

\section*{Computational details}

The results in this paper were obtained using
\proglang{R}~4.1.2 with the
\pkg{RprobitB}~1.0.0.9000 package. \proglang{R} itself
and all packages used are available from the Comprehensive
\proglang{R} Archive Network (CRAN) at \url{https://CRAN.R-project.org/}.


\section*{Acknowledgments}
Content ...

%% -- Bibliography -------------------------------------------------------------
%% - References need to be provided in a .bib BibTeX database.
%% - All references should be made with \cite, \citet, \citep, \citealp etc.
%%   (and never hard-coded). See the FAQ for details.
%% - JSS-specific markup (\proglang, \pkg, \code) should be used in the .bib.
%% - Titles in the .bib should be in title case.
%% - Journal titles should not be abbreviated and in title case.
%% - DOIs should be included where available.
%% - Software should be properly cited as well.

\bibliography{refs}


%% -- Appendix (if any) --------------------------------------------------------
%% - After the bibliography with page break.
%% - With proper section titles and _not_ just "Appendix".

\newpage

\begin{appendix}

\section{Installation} \label{app:installation}

\end{appendix}

%% -----------------------------------------------------------------------------


\end{document}
