\documentclass[article]{jss}

%% -- LaTeX packages and custom commands ---------------------------------------

%% recommended packages
\usepackage{thumbpdf,lmodern}

%% additional packages
\usepackage{amssymb,amsmath}

%% new custom commands
\newcommand{\class}[1]{`\code{#1}'}
\newcommand{\fct}[1]{\code{#1()}}

%% For Sweave-based articles about R packages:
%% need no \usepackage{Sweave}



%% -- Article metainformation (author, title, ...) -----------------------------

%% - \author{} with primary affiliation
%% - \Plainauthor{} without affiliations
%% - Separate authors by \And or \AND (in \author) or by comma (in \Plainauthor).
%% - \AND starts a new line, \And does not.
\author{Lennart Oelschl\"ager \\Bielefeld University \And Dietmar Bauer\\Bielefeld University}
\Plainauthor{Lennart Oelschl\"ager, Dietmar Bauer}

%% - \title{} in title case
%% - \Plaintitle{} without LaTeX markup (if any)
%% - \Shorttitle{} with LaTeX markup (if any), used as running title
\title{\pkg{RprobitB}: Bayes Estimation of Discrete Choice Behavior Heterogeneity via Probit Models in \proglang{R}}
\Plaintitle{RprobitB: Bayes Estimation of Discrete Choice Behavior Heterogeneity via Probit Models in R}
\Shorttitle{RprobitB}

%% - \Abstract{} almost as usual
\Abstract{
\pkg{RprobitB} is an \proglang{R} package for Bayes estimation of probit models with a special focus on modeling choice behavior heterogeneity. In comparison to competing packages it places a focus on approximating the mixing distribution via a latent mixture of Gaussian distributions and thereby providing a classification of deciders. It provides tools for data management, model estimation via Markov Chain Monte Carlo Simulation, diagnostics tools for the Gibbs sampling and a prediction function. This paper demonstrates the functionalities of \pkg{RprobitB} on known choice datasets and compares estimation results across packages.
}

%% - \Keywords{} with LaTeX markup, at least one required
%% - \Plainkeywords{} without LaTeX markup (if necessary)
%% - Should be comma-separated and in sentence case.
\Keywords{discrete choice, probit models, heterogeneity, Bayes estimation, \proglang{R}}
\Plainkeywords{discrete choice, probit models, heterogeneity, Bayes estimation, R}

%% - \Address{} of at least one author
%% - May contain multiple affiliations for each author
%%   (in extra lines, separated by \emph{and}\\).
%% - May contain multiple authors for the same affiliation
%%   (in the same first line, separated by comma).
\Address{
  Lennart Oelschl\"ager\\
  Department of Business Administration and Economics\\
  Bielefeld University\\
  Postfach 10 01 31\\
  E-mail: \email{lennart.oelschlaeger@uni-bielefeld.de}
}

\begin{document}
%% I have no idea what this does. Maybe we need this in the future.
%% \SweaveOpts{concordance=TRUE}

%% -- Introduction -------------------------------------------------------------

%% - In principle "as usual".
%% - But should typically have some discussion of both _software_ and _methods_.
%% - Use \proglang{}, \pkg{}, \fct{} and \code{} markup throughout the manuscript.
%% - If such markup is in (sub)section titles, a plain text version has to be
%%   added as well.
%% - All software mentioned should be properly \cite-d.
%% - All abbreviations should be introduced.
%% - Unless the expansions of abbreviations are proper names (like "Journal
%%   of Statistical Software" above) they should be in sentence case (like
%%   "generalized linear models" below).

\section{Introduction}
\label{sec:intro}

%% Short introduction
The multinomial probit model is one of the most widely-used statistical models to explain the choices that individuals make among a discrete set of alternatives, which is of central interest in many scientific areas, for example in transportation and marketing. In many such choice scenarios it is reasonable to assume, that the preferences of the decision makers are non-homogeneous. Based on personal characteristics, deciders generally weight attributes like time and cost differently. Heterogeneity in choice behavior can be modeled using mixing distributions for the coefficients. Recently, Oelschlaeger and Bauer proposed a new instrument for approximating the underlying mixing distribution that combines Bayes estimation and semi-parametric methods. This paper presents the implementation of the methodology in the \proglang{R} package \pkg{RprobitB}.

%% Overview RUMs
Traditionally, discrete choice models are interpreted as random utility models, including the multinomial logit (MNL) and the multinomial probit (MNP) model as the most prominent members. The MNL model affords straightforward analysis but suffers from the well-known independence of irrelevant alternatives assumption. In contrast, the MNP model avoids this assumption, which however comes at the price of more complex parameter estimation, cf. \cite{Train:09}. In their basic form, these models often fail to take into account heterogeneity of individual deciders, cf. \cite{Train:09}, Chapter 6, or \cite{Train:16}. A concrete example of heterogeneous preferences is constituted by the value of travel time, cf. \cite{Cirillo:06}. Modeling heterogeneity in preferences is indispensable in such cases and has been elaborated in both the MNL and the MNP model by imposing mixing distributions on the coefficients, cf. \cite{Train:09} and \cite{Bhat:11}.

%How are the mixing distribution specified currently?
Specifying these mixing distributions is an important part of the model selection. In absence of alternatives, it has been common practice so far to try different types of standard parametric distributions (including the normal, log-normal, uniform and tent distribution) and to perform a likelihood value-based model selection, cf. \cite{Train:09}, Chapter 6. Aiming to capture correlation patterns across parameters, \cite{Fountas:18} and \cite{Fountas:19} apply multivariate normal mixing distributions in their probit models, which however comes at the price of imposing the rather strong normality assumption on their parameters.

In order to alleviate these restrictions \cite{Train:16} proposes a non-parametric approach based on grid methods. Building on the ideas of \cite{Train:16} and \cite{Bhat:18} recently \cite{Bauer:19} introduced procedures for non-parametrically estimating latent class mixed multinomial probit models where the number of classes is chosen iteratively in the algorithm. These procedures have been demonstrated to be useful in reasonable sized cross-sectional data sets. However, for large panel data sets with a significant number of choice occasions per person, the approach is numerically extremely demanding in particular due to its non-parametric nature and has to deal with the curse of dimensionality.

%What is the potential benefit of Bayesian estimation?
In the Bayesian framework \cite{Scaccia:10} presents the idea to estimate latent class logit models with a fixed prespecified number of Gaussian components. This approach does not require the maximization of the likelihood while at the same time it allows for approximation of the underlying mixing distribution. The same idea has also been applied to probit models, cf. \cite{Xiong:13} for an analysis of adolescent driver-injury data. In both cases however, the specification of the number of latent classes is based only on a trial-and-error strategy.

%What approach do we suggest to specify the mixing distributions?
Oelschlaeger and Bauer presents a more flexible approach that combines the ideas of a Bayesian framework, approximating the mixing distribution through a mixture of normal distributions and updates on the number of latent classes within the algorithm analogously to \cite{Bauer:19}. As a consequence, the procedure unites the benefits of a reduced numerical complexity for the estimation compared to the non-parametric likelihood maximization approach and the ability to approximate any mixing distribution. Presenting simulation results on artificial test cases, it is shown that the approach is capable of approximating the underlying mixing distributions and thereby guiding the specification of mixing distributions for real-world applications.

%% Comparison to other packages
This packages adds to the line of discrete choice software packages in R in the following way: Its focus is entirely on Bayesian estimation, thereby it differs from the packages Rchoice. Furthermore, it places a focus on modeling choice behaviour heterogeneity by approximating the underlying mixing distribution through a latent mixture of normal distributions. The method is explained in detail in Oelschlaeger and Bauer.

%% Article overview
In this article we present the methodology, give an overview over the functionality of the package and apply the package to datasets. Some of them were already analysed and we aim to reconstruct their findings. In addition, we added two datasets that are especially appropriate for the RprobitB package in modeling choice behaviour heterogeneits. The first one is a dataset of contraception choice from the German family panel pairfam. It contains repeated observations of males and femals over several years having different social demographics and relationship status choosing different means of contraception. This choice a priori can be considered to be very hetereogenous and dependent on factors not directly observable by the researcher. The second application deals with the opening choice of chess players depending on their and their openents playing strenght measured in the popular measure system Elo, their gender and nationality. Like the contraception example, this choice a priori can be considers to depend on psychological factors that are not directly observable by the researcher. By applying the functionality of this package we demonstrate how we are able to classify the players into different categories of playing style.

\section{The method} \label{sec:method}



\section{Applications} \label{sec:applications}


\subsection{Choice of contraception} \label{subsec:contraception}


\subsection{Chess opening} \label{subsec:chess}


\section{Conclusion} \label{sec:conclusion}

%% The problem
This paper addressed the problem of specifying mixing distributions in the multinomial probit model with a panel data setting, constituting an important part of the model selection for which the literature does not provide much guidance so far. In the absence of alternatives, many applications of the mixed multinomial probit model rely on different types of standard parametric distributions for modelling heterogeneity in preferences in combination with a likelihood-value based model selection. This course of action is very restrictive and imposes strong assumptions on the distribution of the model parameters that could potentially lead to misspecification and biased estimates.

%% Our solution
We proposed a new approach that improves the current specification strategies in several ways: First, our approach does not require any distributional assumption, since the latent class setting is flexible enough to approximate practically any distribution shape and allowing for any correlation pattern. Second, the weight-based updating scheme ensures that the number of latent classes does not need to be prespecified. Third, the imposed Bayesian estimation framework avoids many numerical problems that occur in the frequentist approach. Most notably, no likelihood function has to be evaluated nor approximated. Comparing the numerical estimation speed to the non-parametric frequentist approach of \cite{Bauer:19}, we found that our implementation of the Bayesian approach is at least 10 times faster. The improvement becomes more distinct for panel data settings with a high number of choice occasions. This is due to the fact that for given total sample size $NT$ a large $T$ is beneficial for the Bayesian approach as then the number of vectors $\beta_n,~ n = 1,...,N$ is comparatively small, while in the frequentist approach calculating the likelihood becomes more challenging for increasing the number $T$ of choice situations faced by each of the $N$ individuals. On the other hand, the grid based frequentist approach of \cite{Bauer:19} can potentially achieve a better approximation (especially of point masses) due to the relatively high number of latent classes. However, this approach requires that a suitable grid is set prior to the estimation with a specification of upper bounds for the coefficients. Additionally, the curse of dimensionality plays a crucial role, which is less of a burden in the Bayesian approach. Note that for a fully specified parametric structure these concerns do not play such a big role also for the frequentist approach.

%% Planned extensions
Our simulation results verified that the proposed approach achieves good approximations of the mixing distribution in common choice modelling situations, where the underlying heterogeneity cannot be captured by standard parametric distributions. It would be interesting to apply the approach also to empirical data in the future. Additionally,
further research on how to properly address sign-restricted coefficients is required.


\section*{Computational details}

The results in this paper were obtained using
\proglang{R}~4.1.3 with the
\pkg{RprobitB}~1.0.0.9000 package. \proglang{R} itself
and all packages used are available from the Comprehensive
\proglang{R} Archive Network (CRAN) at \url{https://CRAN.R-project.org/}.


\section*{Acknowledgments}

This work has been financed partly by the Deutsche Forschungsgemeinschaft (DFG, German Research Foundation) - Projektnummer 356500581 which is gratefully acknowledged.

\bibliography{refs}


\end{document}
